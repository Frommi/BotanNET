% Стрелочки
\def\la{\leftarrow} % <--
\def\ra{\rightarrow} % -->
\def\lra{\leftrightarrow} % <-->
\def\La{\Leftarrow} % <=
\def\Ra{\Rightarrow} % =>
\def\Lra{\Leftrightarrow} % <=>

\def\lrh{\leftrightharpoons}
\def\btu{\bigtriangleup}

% Множества чисел
\def\N{\mathbb{N}}
\def\Z{\mathbb{Z}}
\def\Q{\mathbb{Q}}
\def\R{\mathbb{R}}
\def\C{\mathbb{C}}
\def\H{\mathbb{H}}

\def\LraDef{\stackrel{\mathrm{Def}}{\Lra}}
\def\eqDef{\stackrel{\mathrm{Def}}{=}}

% Меньше или равно, больше или равно
\renewcommand{\le}{\leqslant}
\renewcommand{\ge}{\geqslant}
\renewcommand{\leq}{\leqslant}
\renewcommand{\geq}{\geqslant}

% Математические операторы
\DeclareMathOperator{\diag}{diag}
\DeclareMathOperator{\Int}{int}
\DeclareMathOperator{\cl}{cl}
\DeclareMathOperator{\diam}{diam}
\DeclareMathOperator{\Dom}{Dom}
\DeclareMathOperator{\coDom}{coDom}
\DeclareMathOperator{\Char}{char}
\DeclareMathOperator{\Arg}{Arg}
\DeclareMathOperator{\grad}{grad}
\DeclareMathOperator{\rot}{rot}
\DeclareMathOperator{\divergence}{div}
\DeclareMathOperator{\rang}{rang}
\DeclareMathOperator{\End}{End}
\DeclareMathOperator{\Ann}{Ann}
\AtBeginDocument{\let\Re\relax}
\AtBeginDocument{\newcommand{\Re}{\mathop{\mathrm{Re}}\nolimits}}
\AtBeginDocument{\let\Im\relax}
\AtBeginDocument{\newcommand{\Im}{\mathop{\mathrm{Im}}\nolimits}}
\newcommand{\emod}[1]{\mathop{\equiv}\limits_{#1}}
\newcommand{\Choose}[2]{{\left(#1 \atop #2\right)}}

\newcommand{\partd}[2]{\frac{\partial #1}{\partial #2}}
\newcommand{\partdd}[3]{\frac{\partial^2 #1}{\partial #2\partial #3}}

\let\emptyset\varnothing % красивый emptyset
\newcommand{\eps}{\varepsilon} % красивый epsilon
\newcommand{\sfrac}[2]{{\scriptstyle\frac{#1}{#2}}} % маленькая дробь
\renewcommand{\O}{\mathcal{O}} % O-большое

% Мелочи
\newcommand{\q}[1]{\langle #1 \rangle} % треугольные скобки
\newcommand{\URL}[1]{{\footnotesize{\url{#1}}}} % кликабельная ссылка

% Отступы
\def\makeparindent{\hspace*{\parindent}}
\def\up{\vspace*{-\baselineskip}}
\def\down{\vspace*{\baselineskip}}
\def\LINE{\vspace*{-1em}\noindent \underline{\hbox to 1\textwidth{{ } \hfil{ } \hfil{ } }}}

% Код с правильными отступами
\newenvironment{code}{
  \VerbatimEnvironment

  \vspace{-0.8em}
  \begin{minted}{c++}}{
  \end{minted}
  \vspace{-0.8em}
}

% Формула с правильными отступами
\newenvironment{formula}{
 
  \vspace{-1.0em}
}{
  \vspace{-1.7em}
 
}

\theoremstyle{definition}
\newtheorem{theorem}{Теорема}[section]
\newtheorem{assertion}{Утверждение}[section]
\newtheorem{lemma}{Лемма}[section]

\newtheorem{Def}{$\mathfrak{Def}$}[section]

\theoremstyle{remark}
\newtheorem{conseq}{Следствие}[theorem]
\newtheorem{conseq*}{Следствие}
\newtheorem{Rem}{Замечание}[section]
\newtheorem{exmp}{Пример}[section]
\newtheorem{Exercise}{Упражнение}[section]

\newcommand{\proofbegin}{$\blacktriangleright$}
\newcommand{\proofend}{$\blacktriangleleft$}

% http://tex.stackexchange.com/a/67740
\renewenvironment{proof}{%
	\begin{adjustwidth}{0pt}{\widthof{\proofend}}%
	\begin{description}[labelwidth=\widthof{\proofbegin},leftmargin=!]%
	\item[\proofbegin]%
}{%
    \hfill\makebox[0pt][l]{\proofend} 
	\end{description}%
	\end{adjustwidth}%
}
