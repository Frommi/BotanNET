\documentclass[12pt,a4paper,notitlepage]{article}

\usepackage{subfiles}

\usepackage{fancyhdr}
\usepackage{cmap}
\usepackage[T2A]{fontenc}
\usepackage[utf8]{inputenc}
\usepackage[russian]{babel}
\usepackage{amsthm,amsmath,amssymb,mathtools}
\usepackage[russian]{hyperref}
\usepackage{enumerate}
\usepackage{datetime}
\usepackage{minted}
\usepackage{fancyhdr}
\usepackage{lastpage}
\usepackage{cancel}
\usepackage{pgfplots}
\usepackage[toctitles]{titlesec}
\usepackage{enumitem}
\usepackage{changepage}
\usepackage{wrapfig}
\usepackage{array}
\usepackage{graphicx}
\usepackage{biblatex}

\ifdefined\GeometryForAbstract%
\usepackage[top=0.5cm,bottom=0.5cm,left=1cm,right=1cm,marginparwidth=0.5cm,marginparsep=0cm,headsep=0cm,includehead,includefoot]{geometry}%
\else%
\usepackage[left=1cm,right=1cm,top=2cm,bottom=2cm]{geometry}%
\fi

\pgfplotsset{compat=1.5}


\frenchspacing
\setcounter{MaxMatrixCols}{40}

\gdef\LectureName{}

\setcounter{tocdepth}{2}
\AtBeginDocument{\newcommand{\BigHeader}[3]{%
	\thispagestyle{plain}
	\begin{center}
	\textbf{\Huge #1} \\
	\vspace{1.5em}
	\textbf{\large #2} \\
	\vspace{0.5em}
	\textbf{\large #3} \\
	\vspace{0.5em}
	{Собрано: \today~\currenttime} \\
	\end{center}

	\gdef\LectureName{#1}

    \hrulefill

    {
    \let\clearpage\relax
    %\tableofcontents
    }
    %\clearpage
}}

\setlength{\headheight}{15pt}

\pagestyle{fancy}
\fancyhf{}

\fancyhead{}
\fancyfoot{}

\renewcommand{\headrulewidth}{0.4pt}
\renewcommand{\footrulewidth}{0.4pt}

\renewcommand{\sectionmark}[1]{\markboth{#1}{}}

\fancyfoot[LH]{\LectureName}
\fancyfoot[LF]{МФТИ ФИВТ'20}
\fancyfoot[RH]{\nouppercase{\leftmark}}
\fancyfoot[C]{\thepage~из~\pageref*{LastPage}}
\makeatletter
\fancyfoot[RF]{\normalsize\rmfamily\@ifundefined{currentauthor}{}{Автор: \currentauthor}}
\makeatother

\newcommand{\setauthor}[1]{\gdef\currentauthor{#1}}

% Стрелочки
\def\la{\leftarrow} % <--
\def\ra{\rightarrow} % -->
\def\lra{\leftrightarrow} % <-->
\def\La{\Leftarrow} % <=
\def\Ra{\Rightarrow} % =>
\def\Lra{\Leftrightarrow} % <=>

\def\lrh{\leftrightharpoons}
\def\btu{\bigtriangleup}

% Множества чисел
\def\N{\mathbb{N}}
\def\Z{\mathbb{Z}}
\def\Q{\mathbb{Q}}
\def\R{\mathbb{R}}
\def\C{\mathbb{C}}
\def\H{\mathbb{H}}

\def\LraDef{\stackrel{\mathrm{Def}}{\Lra}}
\def\eqDef{\stackrel{\mathrm{Def}}{=}}

% Меньше или равно, больше или равно
\renewcommand{\le}{\leqslant}
\renewcommand{\ge}{\geqslant}
\renewcommand{\leq}{\leqslant}
\renewcommand{\geq}{\geqslant}

% Математические операторы
\DeclareMathOperator{\diag}{diag}
\DeclareMathOperator{\Int}{int}
\DeclareMathOperator{\cl}{cl}
\DeclareMathOperator{\diam}{diam}
\DeclareMathOperator{\Dom}{Dom}
\DeclareMathOperator{\coDom}{coDom}
\DeclareMathOperator{\Char}{char}
\DeclareMathOperator{\Arg}{Arg}
\DeclareMathOperator{\grad}{grad}
\DeclareMathOperator{\rot}{rot}
\DeclareMathOperator{\divergence}{div}
\DeclareMathOperator{\rang}{rang}
\DeclareMathOperator{\End}{End}
\DeclareMathOperator{\Ann}{Ann}
\AtBeginDocument{\let\Re\relax}
\AtBeginDocument{\newcommand{\Re}{\mathop{\mathrm{Re}}\nolimits}}
\AtBeginDocument{\let\Im\relax}
\AtBeginDocument{\newcommand{\Im}{\mathop{\mathrm{Im}}\nolimits}}
\newcommand{\emod}[1]{\mathop{\equiv}\limits_{#1}}
\newcommand{\Choose}[2]{{\left(#1 \atop #2\right)}}

\newcommand{\partd}[2]{\frac{\partial #1}{\partial #2}}
\newcommand{\partdd}[3]{\frac{\partial^2 #1}{\partial #2\partial #3}}

\let\emptyset\varnothing % красивый emptyset
\newcommand{\eps}{\varepsilon} % красивый epsilon
\newcommand{\sfrac}[2]{{\scriptstyle\frac{#1}{#2}}} % маленькая дробь
\renewcommand{\O}{\mathcal{O}} % O-большое

% Мелочи
\newcommand{\q}[1]{\langle #1 \rangle} % треугольные скобки
\newcommand{\URL}[1]{{\footnotesize{\url{#1}}}} % кликабельная ссылка

% Отступы
\def\makeparindent{\hspace*{\parindent}}
\def\up{\vspace*{-\baselineskip}}
\def\down{\vspace*{\baselineskip}}
\def\LINE{\vspace*{-1em}\noindent \underline{\hbox to 1\textwidth{{ } \hfil{ } \hfil{ } }}}

% Код с правильными отступами
\newenvironment{code}{
  \VerbatimEnvironment

  \vspace{-0.8em}
  \begin{minted}{c++}}{
  \end{minted}
  \vspace{-0.8em}
}

% Формула с правильными отступами
\newenvironment{formula}{
 
  \vspace{-1.0em}
}{
  \vspace{-1.7em}
 
}

\theoremstyle{definition}
\newtheorem{theorem}{Теорема}[section]
\newtheorem{assertion}{Утверждение}[section]
\newtheorem{lemma}{Лемма}[section]

\newtheorem{Def}{$\mathfrak{Def}$}[section]

\theoremstyle{remark}
\newtheorem{conseq}{Следствие}[theorem]
\newtheorem{conseq*}{Следствие}
\newtheorem{Rem}{Замечание}[section]
\newtheorem{exmp}{Пример}[section]
\newtheorem{Exercise}{Упражнение}[section]

\newcommand{\proofbegin}{$\blacktriangleright$}
\newcommand{\proofend}{$\blacktriangleleft$}

% http://tex.stackexchange.com/a/67740
\renewenvironment{proof}{%
	\begin{adjustwidth}{0pt}{\widthof{\proofend}}%
	\begin{description}[labelwidth=\widthof{\proofbegin},leftmargin=!]%
	\item[\proofbegin]%
}{%
    \hfill\makebox[0pt][l]{\proofend} 
	\end{description}%
	\end{adjustwidth}%
}


\begin{document}

\BigHeader
{Проект по БД}
{Техническое задание BotanNET}
{Лиференко~Даниил, Гребенщиков~Иван}

\section{Общее описание}

Сервис для поиска партнеров по боту.
Можно искать как тех, кто ещё не сделал те же задания, что и ты, чтобы вместе их поботать.
Так и искать тех, кто уже сделал задания, чтобы они тебе расшарили.

Программа минимум: консольное приложение на языке Python, принимающая запросы всех пользователей.

Программа максимум: Сервер на Python и web-приложение.

\subsection{Сущности}
В приложении используются такие понятия: юзеры, группы, проекты, таски.

\begin{enumerate}
\item Юзеры "--- пользователи приложения.
\item Группы "--- учебные группы. Сюда входят как и группы типа ``696'', так и ``продвинутый поток по программированию''.
\item Проекты "--- задания, заданные группе. Например, лабораторная по БД.
\item Таски "--- отдельная задача в задании.
\item Комментарии к таскам "--- юзеры могут комментировать таски.
\end{enumerate}

\section{User Stories}
Для любого юзера:
\begin{enumerate}
\item Создание нового пользователя.
\item Изменение личной информации пользователя.
\item Создание новой группы.
\item Добавление/удаление себя из группы.
\item Отметка, что ты выполнил таску (развыполнил).
\item Поиск юзеров, выполнившее или не выполнившие определенные таски по шаблону.
\item Фильтрация результата поиска по группе, этажу, факультету, курсу.
\item Отметка ``не беспокоить'' на таск "--- запретить другим юзерам показывать в запросе с этим таском.
\item Открытие всех комментариев к конкретному таску в хронологическом порядке.
\item Добавление своего комментария в конец списка комментариев для таска.
\end{enumerate}
\newpage
Для админа группы:
\begin{enumerate}
\item Повысить одногруппника до админа группы.
\item Создать проект в группе.
\item Добавить таск в проект.
\end{enumerate}

\section{Схема базы данных}

В базе данных будет $n$ таблиц:
\begin{enumerate}
\item \texttt{Users}
    \begin{enumerate}
    \item \texttt{user\_id, serial}
    \item \texttt{user\_name, varchar(50)}
    \item \texttt{department, varchar(30)}
    \item \texttt{dorm\_number, int}
    \item \texttt{group\_number, int}
    \end{enumerate}
\item Groups
    \begin{enumerate}
    \item \texttt{group\_id, serial}
    \item \texttt{group\_name, varchar(50)}
    \item \texttt{teacher\_name, varchar(100)}
    \end{enumerate}
\item Projects
    \begin{enumerate}
    \item \texttt{project\_id, serial}
    \item \texttt{group\_id, int}
    \item \texttt{project\_name, varchar(50)}
    \item \texttt{project\_info, text}
    \item \texttt{deadline, timestamp}
    \end{enumerate}
\item Tasks
    \begin{enumerate}
    \item \texttt{task\_id, serial}
    \item \texttt{task\_info, text}
    \end{enumerate}
\item UserGroups
    \begin{enumerate}
    \item \texttt{user\_id, int}
    \item \texttt{group\_id, int}
    \item \texttt{is\_admin, bool}
    \end{enumerate}
\item ProjectTasks
    \begin{enumerate}
    \item \texttt{project\_id, int}
    \item \texttt{task\_id, int}
    \end{enumerate}
\item UserTasks
    \begin{enumerate}
    \item \texttt{user\_id, int}
    \item \texttt{task\_id, int}
    \item \texttt{is\_done, bool}
    \item \texttt{is\_hidden, bool}
    \end{enumerate}
\item Comments
    \begin{enumerate}
    \item \texttt{task\_id, int}
    \item \texttt{comment\_no, int}
    \item \texttt{user\_id, int}
    \item \texttt{comment\_text, text}
    \end{enumerate}
\end{enumerate}

\end{document}
